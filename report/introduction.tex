\section{Introduction}

Recent advances in natural language processing techniques alongside the development of the field of Digital Humanities have lead to explorations of questions revolving around literary narrative. It is in this context that we investigate approaches to automatic extraction of metadata focusing on characters in French 19th century novels.

Natural language processing (NLP), as a discipline has found applications in ever wider domains. Since the expression and comprehension of thoughts and experiences in language is one of the foremost means of human communication, it is a capacity essential to understanding our wants and needs. As we develop the methods relevant to this discipline, we are able to further the complexity and depth of interaction between humans and machines. 

Digital Humanities as a field of endeavour is pushing boundaries by ``bringing the tools and techniques of digital media to bear on traditional humanistic questions'' \footnote{Kathleen Fitzpatrick \href{http://www.inthelibrarywiththeleadpipe.org/2015/on-scholarly-communication-and-the-digital-humanities-an-interview-with-kathleen-fitzpatrick/}{"On Scholarly Communication and the Digital Humanities: An Interview with Kathleen Fitzpatrick"}, In the Library with the Lead Pipe}. The field is striving to develop the complementarity to two rich and thriving aspects of human civilization. 

These humanistic questions, in our case, mean the study of literature. Our focus is specifically on the linguistic patterns of the elucidation of meta-information of characters in French 19th century literature, but many of the hypotheses, assumptions, and findings can be generalized to the wider field of fictional, and possibly non-fictional, character description. 

There have been many advances in the automatic study of fictional and non-fictional personas through natural language processing. This ranges from entity-centric \cite{chambers2013event} to event-centric \cite{cheung2013probabilistic} approaches. The persona is analyzed from varying standpoints, be it the role of the character within a particular fictional narrative \cite{valls2014toward}, or linking the observed character traits to non-fictional situations, like personality profiling \cite{flekova2015personality}, closely tied to that very practice in modern psychology.

All of the insights generated through these studies enrich our understanding of how narrative is developed, with regard to a character. The approach presented here focuses on yet another aspect of the fictional literary character, i.e. the social and behavioral elements tied to a characters existence, with the hope of broadening still this understanding. Some examples of these elements might be a character's professional occupation, place of origin, social status, age, gender. In short, elements that increase the knowledge of the character within its social existence, as we perceive it in the non-fictional sense. Throughout this work, we refer to these aspects as a character's \textit{metadata}.

Since our focus in this instance is specifically on French literature, it has some unavoidable implications on the procedure undertaken. Mainly, the predominant tools and procedures having been and being developed in all fields of NLP are focused on the English language. There are of course developments and resources in other languages, but it is as such not a space where the same resources are readily available (think coreference resolution, dependency parsing, etc.). It is worth noting that the relative sparsity of work done with regard to French (and other) language processing and the particularities of each individual language make such work interesting. 

As a result, the approaches tried in our case focus mainly on shallow NLP ideas. These tend to ``not attempt to achieve an exhaustive linguistic analysis'', and ``are designed for specific tasks ignoring many details in input and linguistic (grammar) framework'' \cite{schafer2007integrating}. There is also a distinct advantage in such an approach, in that they often allow for less time- and resource-consuming methods. 

This work is in no way exhaustive in the possibilities it presents, nor the opportunities explored. It does however aim to present how feasible a shallow approach to fiction is for understanding specific aspects of characters, by empirical measures of the efficiency of the methods explained below.